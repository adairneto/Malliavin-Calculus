\chapter{Application to Finance}

\section{Introduction}

Getting back to the context of the section \hyperref[portfolio-selection]{Portfolio Selection}, we saw that, for a contingent claim with payoff $F$ (where $F \in L^2(\P)$, is $\mathcal{F}_T$-measurable and bounded below), the replicating portfolio is given by \eqref{eq:202405241112}. In particular, if $\rho$ is constant, and $\mu$ and $\sigma$ are deterministic, then we have \eqref{eq:202406031637}, which we reproduce here for the reader's ease.
\[
\theta_1 (t) = e^{-\rho(T-t)} \sigma^{-1}(t) S_1^{-1}(t) \E_Q [D_t F \mid \mathcal{F}_t]
\]

In this chapter, we consider a digital option, which has a payoff at maturity
\[
F = \mathbf{1}_{[K, \infty)}(W(T))
\]
where $K$ is the exercise price. We aim to compute the conditional expectation $\E_Q [D_t F \mid \mathcal{F}_t]$. 

\section{Necessary Results}

To do that, we need the following concept.

\begin{definition}[Donsker delta function]
Let $Y : \Omega \longrightarrow \R$, $Y \in \mathcal{G}^\ast$. The continuous function
\[
\delta_Y (\cdot) : \R \longrightarrow {\mathcal{G}^{\ast}}
\]
is the \textbf{Donsker delta function} of $Y$ if it has the property that
\[
\int_\R f(y) \delta_Y(y) ~\mathrm{d}y = f(Y) \quad \text{a.s.}
\]
for all measurable $f : \R \longrightarrow \R$ such that the integral converges. 
\end{definition}

\begin{theorem}\label{thm:aase4.4}
Suppose that
\begin{enumerate}
\item $\alpha : [0,T] \longrightarrow \R^n$ is a deterministic function such that $\| \alpha \|^2 = \int_0^T \alpha^2(s) ~\mathrm{d}s < \infty$.
\item $\varphi : [0,T] \longrightarrow \R^{n \times n}$ is a deterministic function such that $\| \varphi \|^2 = \sum_{i,j=1}^n \int_0^T \varphi_{ij}^2(s) ~\mathrm{d}s < \infty$.
\item $f : \R^n \longrightarrow \R$ is bounded.
\end{enumerate} 

Define, for $t \in [0,T]$,
\begin{equation*}
Y(t) = \int_0^t \varphi(s) ~\mathrm{d}B(s) + \int_0^t \varphi(s) \alpha(s) ~\mathrm{d}s
\end{equation*}

Then 
\begin{equation*}
f(Y(T)) = V_0 + \int_0^T u(t, \omega) \diamond (\alpha(t) + W(t)) ~\mathrm{d}t
\end{equation*}
where $|A| = \det A$ and $A$ is the inverse of the covariance matrix of $Y$,
\begin{equation*}
V_0 = (2 \pi)^{-n/2} \sqrt{|A|} \int_{\R^n} f(y) \exp \left( -\frac{1}{2} y^T Ay \right) ~\mathrm{d}y
\end{equation*}
and 
\begin{equation*}
u(t, \omega) = (2 \pi)^{-n/2} \sqrt{|A|} \int_{\R^n} f(y) \exp^\diamond \left( -\frac{1}{2} (y-Y(t))^T \diamond A(y-Y(t)) \right) \diamond ((y-Y(t))^T A \varphi(t)) ~\mathrm{d}y
\end{equation*}
\end{theorem}

\begin{proof}
Refer to \cite[Theorem 4.4]{aase2001using}.
\end{proof}

The next result is a simpler version of \cite[Lemma 3.21]{hu2003fractional}.

\begin{theorem}\label{thm:wick-diff-meas}
Let $\diamond_\P$ and $\diamond_Q$ denote the Wick product for the probability measures $\P$ and $Q$ respectively, and $u$ as in the \hyperref[thm:girsanov]{Girsanov Theorem}. Then, $F \diamond_\P G = F \diamond_Q G$.
\end{theorem} 

\begin{proof}
Let $F = \exp^\diamond \left( \int_0^\infty f(t) ~\mathrm{d}W(t) \right)$ and $G = \exp^\diamond \left( \int_0^\infty g(t) ~\mathrm{d}W(t) \right)$. Then, 
\begin{equation*}
\begin{aligned}
F \diamond_\P G = \exp \left( \int_0^\infty (f(s) + g(s)) ~\mathrm{d}W(s) - \frac{1}{2} \|f + g\|^2 \right)
\end{aligned}
\end{equation*}

Applying Girsanov to $F$ and $G$,
\[
F = \exp^\diamond \left( \int_0^\infty f(s) ~\mathrm{d}\widetilde{W}(s) - \langle f, u \rangle_{L^2} \right), 
\quad \text{ and } \quad
G = \exp^\diamond \left( \int_0^\infty g(s) ~\mathrm{d}\widetilde{W}(s) - \langle g, u \rangle_{L^2} \right) 
\]

Computing the product,
\begin{equation*}
\begin{aligned}
F \diamond_Q G &= \exp \left( \int_0^\infty (f(s) + g(s)) ~\mathrm{d}\widetilde{W}(s) - \frac{1}{2} \|f + g\|^2 - \langle f+g, u \rangle_{L^2} \right) \\
&= \exp \left( \int_0^\infty (f(s) + g(s)) ~\mathrm{d}W(s) - \frac{1}{2} \|f + g\|^2 \right)
\end{aligned}
\end{equation*}

Hence, $F \diamond_\P G = F \diamond_Q G$ for exponential functions. By density, we have the result.
\end{proof}

\section{Main Result}

\begin{theorem}
Suppose that $\rho$ is constant, and $u(t)$, as defined in \eqref{eq:202401171041}, is deterministic and satisfying $\E [u^2(t)] < \infty$. Then the replicating portfolio for hedging $\mathbf{1}_{[K, \infty)}(W(T))$ is 
\begin{equation}\label{eq:replicating-digital}
\theta_1(t) = e^{-\rho(T-t)} (2 \pi (T-t))^{-1/2} \sigma^{-1}(t) S_1^{-1}(t) \exp \left( - \frac{(K - W(t))^2}{2(T-t)} \right)
\end{equation}
\end{theorem}

\begin{proof}
First, notice that $F = \mathbf{1}_{[K, \infty)}(W(T)) \in L^2(\P)$. Thus, we can use \eqref{eq:202406031637}.

Now we compute $\E_Q [D_t F \mid \mathcal{F}_t]$ using the Donsker delta function by taking $f(y) = \mathbf{1}_{[K, \infty)}(y)$, and $Y(T) = W(T)$. By the Theorem \ref{thm:aase4.4},
\[
\mathbf{1}_{[K, \infty)}(W(T)) = \int_K^\infty (2 \pi T)^{-1/2} \exp^\diamond \left( - \frac{(y-W(T))^{\diamond 2}}{2T} \right) ~\mathrm{d}y
\]

By the \hyperref[thm:wick-chain]{Chain Rule for the Wick product}, 
\begin{equation*}
\begin{aligned}
D_t (\mathbf{1}_{[K, \infty)}(W(T))) &= \int_K^\infty (2 \pi T)^{-1/2} \exp^\diamond \left( - \frac{(y-W(T))^{\diamond 2}}{2T} \right) \diamond \frac{(y-W(T))}{2T} ~\mathrm{d}y \\ 
&= (2 \pi T)^{-1/2} \exp^\diamond \left( - \frac{(K-W(T))^{\diamond 2}}{2T} \right)
\end{aligned}
\end{equation*}

Denoting by $\hat{\diamond}$ the Wick product with respect to the probability measure $Q$, then since $\hat{\diamond} = \diamond$ (Theorem \ref{thm:wick-diff-meas}), we have

\begin{equation*}
\begin{aligned}
\E [D_t (\mathbf{1}_{[K, \infty)}(W(T))) \mid \mathcal{F}_t] &= \E_Q \left.\left[ (2\pi T)^{-1/2} \exp^{\hat{\diamond}} \left(- \frac{(K-W(T))^{\hat{\diamond} 2}}{2T} \right) ~ \right| ~ \mathcal{F}_t \right] \\
&= (2\pi T)^{-1/2} \E_Q \left.\left[ \exp^{\hat{\diamond}} \left(- \frac{(K-\widetilde{W}(T) + \int_0^T u(s) ~\mathrm{d}s)^{\hat{\diamond} 2}}{2T} \right) ~ \right| ~ \mathcal{F}_t \right] \\
&= (2\pi T)^{-1/2} \exp \left( - \frac{(K-W(t))^2}{2 (T-t)} \right)
\end{aligned}
\end{equation*}

\end{proof}

