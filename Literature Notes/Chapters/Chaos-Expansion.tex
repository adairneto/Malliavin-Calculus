\chapter{The Wiener-Itô Chaos Expansion}

\section{Iterated Itô Integrals}

Consider

\begin{itemize}
\item
  \(W(t)\) a one-dimensional Wiener process on a complete probability
  space \((\Omega, \mathcal{F}, \mathbb{P})\) such that \(W(0) = 0\)
  a.s.;
\item
  \(\mathcal{F}_t\) the \(\sigma\)-algebra generated by \(W(s)\),
  \(0 \le s \le t\), augmented by the events with probability zero;
\item
  The filtration \(\mathbb{F} = \{ \mathcal{F}_t : t \in [0, T] \}\).
\end{itemize}

\begin{definition}
    A function \(g : [0,T]^n \longrightarrow \mathbb{R}\) is \textbf{symmetric} if
    \[
    g(t_{\sigma_1}, \ldots, t_{\sigma_n}) = g(t_1, \ldots, t_n)
    \]
    for all permutations \(\sigma = (\sigma_1, \ldots, \sigma_n)\) of
    \((1, 2, \ldots, n)\).

    We define \(\hat{L}^2([0,T]^n)\) as the subspace of \(L^2([0,T]^n)\)
    consisting of symmetric functions.
\end{definition}

Now consider the set
\[
S_n = \{ (t_1, \ldots, t_n) \in [0, T]^n : 0 \le t_1 \le \cdots \le t_n \le T \}
\]

Notice that the set \(S_t\) occupies \(\frac{1}{n!}\) of the whole box
\([0,T]^n\). Thus, if \(g \in \hat{L}^2([0,T]^n)\), then
\(g|_{S_n} \in L^2(S_n)\) and
\[
\| g \|_{L^2([0,T]^n)}^2 = n! \int_{S_n} g^2(t_1, \ldots, t_n) ~\mathrm{d} t_1 \cdots \mathrm{d}t_n = n! \| g \|_{L^2(S_n)}^2
\]

\begin{definition}
    Given a function \(f : [0,T]^n \longrightarrow \mathbb{R}\), we define its
    \textbf{symmetrization} \(\hat{f}\) as
    \[
    \hat{f}(t_1, \ldots, t_n) = \frac{1}{n!} \sum_{\sigma} f(t_{\sigma_1}, \ldots, t_{\sigma_n})
    \]
    in which the sum runs over all permutations \(\sigma\) of \(\{ 1, 2, \ldots, n \}\).
\end{definition}

Now we are ready to define the \(n\)-fold iterated Itô integral.

\begin{definition}
    Let \(f\) be a deterministic function defined on
    \(S_n\) such that \(\| f \|_{L^2(S_n)}^2 < \infty\). We define the
    \textbf{\(n\)-fold iterated Itô integral} as \[
    J_n(f) = \int_0^T \int_0^{t_n} \cdots \int_0^{t_3} \int_0^{t_2} f(t_1, \ldots, t_n) ~\mathrm{d}W(t_1) \mathrm{d}W(t_2) \cdots \mathrm{d}W(t_{n-1}) \mathrm{d}W(t_n)
    \]
\end{definition}

\begin{remark}
    \begin{enumerate}
        \def\labelenumi{\arabic{enumi}.}
        \item
          Note that each \(i\)-th Itô integral with respect to
          \(\mathrm{d}W(t_i)\) is well-defined, since the integrand is an
          \(\mathbb{F}\)-adapted stochastic process.
        \item
          Furthermore, \(J_n(f) \in L^2(\mathbb{P})\).
    \end{enumerate}
\end{remark}

Now, apply Itô's isometry iteratively (we can apply it because \(g \in L^2(S_m)\) and \(h \in L^2(S_n)\)?).

\emph{First case:} If \(g \in L^2(S_m)\) and \(h \in L^2(S_n)\) with
\(m < n\), then \[
\begin{aligned}
    \mathbb{E}[J_m(g) J_n(h)] &= \mathbb{E} \left[ \left( \int_0^T \int_0^{s_m} \cdots \int_0^{s_2} g(s_1, \ldots, s_m) ~\mathrm{d}W(s_1) \cdots \mathrm{d}W(s_m) \right) \right. \\
    & \left. \left( \int_0^T \int_0^{s_m} \cdots \int_0^{t_2} h(t_1, \ldots, t_{n-m}, s_1, \ldots, s_m) ~\mathrm{d}W(t_1) \cdots \mathrm{d}W(t_{n-m}) \mathrm{d}W(s_1) \cdots \mathrm{d}W(s_m) \right) \right] \\
    &= \int_0^T \int_0^{s_m} \cdots \int_0^{s_2} g(s_1, s_2, \ldots, s_m)  \\
    &\mathbb{E} \left[ \int_0^{s_1} \cdots \int_0^{t_2} h(t_1, \ldots, t_{n-m}, s_1, \ldots, s_m) ~\mathrm{d}W(t_1) \cdots \mathrm{d}W(t_{n-m}) \right] ~\mathrm{d}s_1 \cdots \mathrm{d}s_m \\
    &= 0
\end{aligned}
\]

\emph{Second case:} If \(g, h \in L^2(S_n)\), then \[
\begin{aligned}
    \mathbb{E}[J_n(g) J_n(h)] &= \int_0^T \cdots \int_0^{s_2} g(s_1, \ldots, s_n) h(s_1, \ldots, s_n) ~\mathrm{d}s_1 \cdots \mathrm{d}s_n = \langle g, h \rangle_{L^2(S_n)}
\end{aligned}
\] in which \(\langle g, h \rangle_{L^2(S_n)}\) is the inner product of
\(L^2(S_n)\).

This proves the following

\begin{proposition}
    For \(m, n \in \mathbb{Z}_{>0}\), \[
    \mathbb{E}[J_m(g) J_n(h)] = 
    \begin{cases}
        0, & n \neq m \\
        \langle g, h \rangle_{L^2(S_n)}, & n = m
    \end{cases}
    \] In particular, \[
    \| J_n(h) \|_{L^2(\mathbb{P})} = \| h \|_{L^2(S_n)}
    \] For \(n = 0\) or \(m = 0\), we define \(J_0(g) = g\), when \(g\) is a
    constant, and \(\langle g, h \rangle_{L^2(S_0)} = gh\), when \(g\) and
    \(h\) are constants.
\end{proposition}

\begin{remark}
    \begin{enumerate}
        \def\labelenumi{\arabic{enumi}.}
        \item
          If \(f \in L^2(S_n)\), then \(J_n(f) \in L^2(\mathbb{P})\).
        \item
          The \(n\)-fold iterated Itô integral is a linear operator.
        \end{enumerate}
\end{remark}

\begin{definition}
    Let \(g \in \hat{L}^2([0,T]^n)\). Then \[
    I_n(g) = \int_{[0,T]^n} g(t_1, \ldots, t_n) ~\mathrm{d}W(t_1) \cdots \mathrm{d}W(t_n) = n! J_n(g)
\] is also called \textbf{\(n\)-fold iterated Itô integral}.
\end{definition}

Let $x \in \mathbb{R}$ and $n = 0, 1, 2, \ldots$. Then the \textbf{Hermite polynomials} are defined by
$$
h_n(x) = (-1)^n e^{\frac{1}{2} x^2} \frac{d^n}{dx^n} \left( e^{-\frac{1}{2} x^2} \right)
$$

For example, the first Hermite polynomials are 
\begin{enumerate}
\item $h_0(x) = 1$,
\item $h_1(x) = x$,
\item $h_2(x) = x^2 - 1$,
\item $h_3(x) = x^3 - 3x$, 
\item $h_4(x) = x^4 - 6x^3 + 3$,
\item $h_5(x) = x^5 - 10x^3 + 15x$.
\end{enumerate}

The family of Hermite polynomials constitute an orthogonal basis for $L^2(\mathbb{R}, \mu(dx))$, in which $\mu(dx) = \frac{1}{\sqrt{2 \pi}} e^{\frac{x^2}{2}} dx$.

\begin{proposition}\label{prop:202311031501}
    If \(\xi_1, \xi_2, \ldots\) are orthonormal
    functions in \(L^2([0,T])\), then \[
    I_n(\xi_1^{\otimes \alpha_1} \hat{\otimes} \cdots \hat{\otimes} \xi_m^{\otimes \alpha_m}) = \prod_{k=1}^m h_{\alpha_k} \left( \int_0^T \xi_k(t) W(t) \right)
    \] with \(\alpha_1 + \cdots +\alpha_m = n\),
    \(\alpha_k \in \mathbb{N}_0\) for all \(k\), and \(\hat{\otimes}\) is
    the symmetrized tensor product, which is the symmetrization of
    \(f \otimes g\).
\end{proposition}

\begin{proof}
    \cite{ito1951multiple}
\end{proof}

See Itô's formula for iterated Itô integral
(\cite{oksendal2013stochastic}'s problem 3.7).

Using this, it is possible to prove (see
\cite{nualart2018introduction} p.~64) \[
I_n(g^{\otimes n}) = n! \int_0^T \int_0^{t_n} \cdots \int_0^{t_2} g(t_1) g(t_2) \cdots g(t_n) ~\mathrm{d}W(t_1) \cdots \mathrm{d}W(t_n) = \| g \|^n h_n \left( \frac{\int_0^T g(t) ~\mathrm{d}W(t)}{\| g \|} \right)
\]

\section{The Wiener-Itô Chaos Expansion}

\begin{theorem}[The Wiener-Itô Chaos Expansion]\label{thm:chaos-expansion}
    Let $\xi$ be an $\mathcal{F}_T$-measurable random variable in $L^2(\mathbb{P})$. There exists a unique sequence $(f_n)$ of functions $f_n \in \hat{L}^2([0,T]^n)$ such that
    $$
    \xi = \sum_{n=0}^\infty I_n(f_n)
    $$
    with convergence in $L^2(\mathbb{P})$. Moreover, we have the following isometry:
    $$
    \| \xi \|_{L^2(\mathbb{P})}^2 = \sum_{n=0}^\infty n! \| f_n \|_{L^2([0,T]^n)}^2
    $$
\end{theorem}

\begin{proof}
    The idea is to obtain an orthogonal decomposition of $L^2(\mathbb{P})$. To do that, we show that $\psi$ is orthogonal to $\exp(\theta)$, which form a total set in $L^2(\mathbb{P})$, implying that $\psi \equiv 0$.

    1. Apply Itô's Representation Theorem twice.

    2. Use Itô's Isometry and then Itô's Representation Theorem again.

    3. Iterate this procedure $n+1$ times, we obtain a process $\varphi_{n+1}(t_1, \ldots, t_{n+1})$, $0 \le t_1 \le \cdots \le t_{n+1} \le T$, and $n+1$ deterministic functions $g_0, g_1, \ldots, g_n$ (where $g_0 = \mathbb{E}[\xi]$ and $g_k(s_k, s_{k-1}, \ldots, s_1) = \mathbb{E}[\varphi_k(s_k, s_{k-1}, \ldots, s_1)]$ for $1 \le k \le n$) such that 
        $$
        \xi = \sum_{k=0}^n J_k(g_k) + \int_{S_{n+1}} \varphi_{n+1} ~\mathrm{d}W^{\otimes(n+1)}
        $$

    4. Note that we have a $(n+1)$-fold Iterated Itô Integral
        $$
        \int_{S_{n+1}} \varphi_{n+1} ~\mathrm{d}W^{\otimes(n+1)} =: \psi_{n+1}
        $$
        and 
        $$
        \mathbb{E} \left[ \left( \int_{S_{n+1}} \varphi_{n+1} ~\mathrm{d}W^{\otimes(n+1)} \right)^2 \right] \le \mathbb{E}[\xi^2]
        $$

    5. Also remark that the family $\psi_{n+1}$ is bounded in $L^2(\mathbb{P})$ and, from Itô's Isometry,
        $$
        \langle \psi_{n+1}, J_k(f_k) \rangle_{L^2(\mathbb{P})} = 0
        $$
        for $k \le n$ and $f_k \in L^2([0,T]^k)$

    6. Compute $\| \xi \|_{L^2(\mathbb{P})}^2$ and notice that $\sum_{k=0}^\infty J_k(g_k)$ is convergent in $L^2(\mathbb{P})$. Thus,
        $$
        \langle J_k(f_k), \psi \rangle_{L^2(\mathbb{P})} = 0
        $$

    7. Using that
    $$
    I_n(g^{\otimes n}) = \| g \|^n h_n \left(\frac{\theta}{\| g \|} \right), \quad \theta = \int_0^T g(t) ~\mathrm{d}W(t)
    $$
    and Hermite polynomials, we have
    $$
        \mathbb{E} \left[ h_n \left(\frac{\theta}{\| g \|} \right) \psi \right] = 0, \quad 
        \mathbb{E} \left[ \theta^k \psi \right] = 0, \quad \mathbb{E} \left[ \exp \theta \cdot \psi \right] = \sum_{k=0}^\infty \frac{1}{k!} \mathbb{E} \left[ \theta^k \psi \right] = 0
    $$

    8. Since $\{ \exp \theta : g \in L^2([0,T]^n) \}$ is total in $L^2(\mathbb{P})$ (i.e. its linear span is dense) \cite[Lemma 4.3.2]{oksendal2013stochastic}, $\psi = 0$. Thus, we obtain 
        $$
        \xi = \sum_{k=0}^\infty J_k(g_k)
        $$
        and
        $$
        \| \xi \|_{L^2(\mathbb{P})}^2 = \sum_{k=0}^\infty \| J_k(g_k) \|_{L^2(\mathbb{P})}^2
        $$

    9. To extend $g_n$ from $S_n$ to $[0,T]^n$, we put 
        $$
        g_n(t_1, \ldots t_n) = 0, \quad (t_1, \ldots t_n) \in [0,T]^n \setminus S_n
        $$
        and define $f_n = \hat{g}_n$, i.e., the symmetrization of $g_n$.

        Then, 
        $$
        I_n(f_n) = n! J_n(f_n) = n! J_n(\hat{g}_n) = J_n(g_n)
        $$
        and the result follows.
\end{proof}
